%!TEX root = main.tex


\section{Analysis of Vaidya walk} % (fold)
\label{sec:proof}

In this section, we first provide an outline of a general method of bounding the rate of convergence of geometric random walks on convex sets in Section~\ref{sub:a_general_method_to_bound_mixing_time}, followed by some auxiliary results in Section~\ref{sub:some_auxiliary_results} that are used to invoke the general method in our case.
Using these auxiliary results, we first contrast the differences between Dikin walk and Vaidya walk in Section~\ref{sub:vaidya_vs_dikin}, and then prove our main result in Section~\ref{sub:proof_of_theorem_thm:mixing_time_bound}.
 % and \ref{sub:proof_of_theorem_thm:central_start}.
We provide the proofs of the auxiliary results in Sections~\ref{sub:proof_of_lemma_lemma:lovasz_theorem}, \ref{ssub:proof_of_lemma_lemma:first_bounds} and \ref{ssub:proof_of_lemma_lemma:vaidya_walk_close_kernel}, and defer some technical results to appendices.

\subsection{A general method to bound mixing time} % (fold)
\label{sub:a_general_method_to_bound_mixing_time}

For a discrete-space Markov chain, a bound on mixing time is obtained via bounds on the \emph{spectral gap} of the transition matrix associated with the chain.
Often, an indirect bound on the spectral gap is obtained via Cheeger's inequality that bounds the spectral gap in terms of the \emph{conductance} of the chain.
Lov{\'a}sz and Simonovits~\cite{lovasz1993random} proved a similar connection between conductance and convergence rate for continuous-space Markov chains.
Thus proving an upper bound for the mixing time of a geometric random walk on convex sets often boils down to showing a good lower bound on the conductance of the chain---these arguments have been used for ball walk~\cite{lovasz1990ballwalk}, Hit-and-run~\cite{lovasz1999hit,lovasz2006hit} and Dikin walk~\cite{narayanan2016randomized,kannan2012random,sachdeva2016mixing} on convex sets.
When the underlying space is convex and bounded, using some isoperimetric inequalities, Lovasz showed that to get a lower bound on the conductance of the Markov chain with stationary distribution uniform, it suffices to establish that the chain satisfies the following good-neighborhood-property: ``if two points are close, then their one-step transition distribution are also close.''
The mixing time of the chain roughly scales with the square of the inverse of ``how close the two-points need to be'' in order for their one-step transition distributions to be close.
We show that for Vaidya walk when compared to Dikin walk, the points can be much far apart, with their one-step transition distributions still being close.
Consequently, our walk mixes faster than Dikin walk.
Much of our work focuses on quantifying the last claim formally.
We restrict our attention to formalizing the ``good-neighborhood-property'', and refer the reader to the survey by Vempala~\cite{vempala2005geometric} for a more thorough and formal discussion about the usual techniques and the proofs of other arguments that have been summarized above.

\begin{figure}
  \centering
  \begin{subfigure}[h]{0.45\linewidth}
    \centering
    \includegraphics[width=0.8\linewidth]{figures/polytope}
    \caption{}
    \label{fig:basic_polytope}
  \end{subfigure}
  \begin{subfigure}[h]{0.45\linewidth}
    \centering
    \includegraphics[width=0.8\linewidth]{figures/cross-ratio}
    \caption{}
    \label{fig:cross_ratio}
  \end{subfigure}
  \caption{Polytope $\Pspace = \{x\in \realdim\vert Ax \leq b \}$.
  (a)~The points $e(x)$ and $e(y)$ denote the intersection points of the chord joining $x$ and $y$ with $\Pspace$ such that $e(x), x, y, e(y)$ are in order.
  (b)~A geometric illustration of the argument~\eqref{eq:relate_cross_ratio}. It is straightforward to observe that ${\vecnorm{x-y}{2}}/{\vecnorm{e(x)-x}{2}} = {\vecnorm{u-y}{2}}/{\vecnorm{u-v}{2}} = {\abss{a_i\tp(y-x)}}/\parenth{b_i-a_i\tp x}$.
  }
  \label{fig:polytope_and_end_points}
\end{figure}

We now present a formal statement of the good-neighborhood-property for a random walk on a bounded convex set.
For quantifying closeness in the property, the distributions are contrasted with the total-variation distance, while for distance between points we use the cross ratio, that we define next.
For a given pair of points $x, y \in \Pspace$, let $e(x), e(y) \in \partial \Pspace$ denote the intersection of the chord joining $x$ and $y$ with $\Pspace$ such that $e(x), x, y, e(y)$ are in order (see Figure~\ref{fig:basic_polytope}).
The cross-ratio $d_\Pspace(x, y)$ is given by
\begin{align}
\label{eq:def_cross_ratio}
	d_\Pspace(x, y) = \frac{ \vecnorm{e(x)-e(y)}{2} \vecnorm{x-y}{2} }{ \vecnorm{e(x)-x}{2} \vecnorm{e(y)-y}{2} }.
\end{align}
The ratio $d_\Pspace(x, y)$ is related to the Hilbert metric on $\Pspace$, which is given by $\log \parenth{ 1+d_\Pspace(x, y)}$ (see the paper by Bushell \cite{bushell1973hilbert} for more details).

Let $X_0, X_1, \ldots $ denote a \emph{lazy} reversible random walk on a bounded convex set $\Pspace$ with transition kernel $\lazytrans :x\mapsto \delta_x(\cdot)/2 + \nolazytrans_x(\cdot)/2$ and stationary with respect to the uniform distribution on $\Pspace$ (denoted by $\stationary$).
The following lemma gives a bound on the mixing-time of the chain $\{X_t, t\geq 0\}$.
\begin{lemma}
	\label{lemma:Lovasz_theorem}
	Suppose that $\vecnorm{\nolazytrans_x- \nolazytrans_y}{\text{TV}} \leq 1 - \rho$, for all $x, y \in \intP$ such that $d_\Pspace(x, y)~< \Delta $, for some $\rho,\Delta \in (0, 1)$.
	Then for every distribution $\initial$ that is $M$-warm with respect to $\stationary$, we have
	\begin{align*}
		\vecnorm{\initial \lazytrans^k-\stationary}{\text{TV}} \leq \displaystyle\sqrt{M}\exp\parenth{-k\,\frac{\Delta^2 \rho^2}{512}}.
	\end{align*}
\end{lemma}
The proof of the lemma follows from an isoperimetry inequality involving cross-ratio~\cite{lovasz1999hit} and the classical mixing time bound in terms of conductance~\cite{lovasz1993random}.
Similar results have also been used in the proofs of Hit-and-run and Dikin walk.
We provide the proof of the lemma in Section~\ref{sub:proof_of_lemma_lemma:lovasz_theorem}.
To prove Theorem~\ref{thm:mixing_time_bound}, we show that the random walk \VW{1/9000} satisfies the assumptions of Lemma~\ref{lemma:Lovasz_theorem} with suitable $\Delta$ and $\rho$ which yields the claimed mixing time bound.
Besides Lemma~\ref{lemma:Lovasz_theorem}, our proof techniques are inspired by the proofs of convergence rate of Dikin walk on polytopes presented by Kannan~\etal~\cite{kannan2012random}
and the simple proof of Dikin walk provided by Sachdeva~\etal~\cite{sachdeva2016mixing}.

% subsection a_general_method_to_bound_mixing_time (end)

\subsection{Some auxiliary results} % (fold)
\label{sub:some_auxiliary_results}
We now introduce some notation and auxiliary results that will be useful for the proof.
For all $x \in \Pspace$ let $\slack_{x} \defn (b_1-a_1\tp x, \ldots, b_\obs - a_\obs\tp x)\tp$ denote the ``slackness at $x$''.
For all $x\in \intP$, define the ``local norm at $x$'' as
\begin{subequations}
\begin{align}
	\vecnorm{.}{x} : v \mapsto \vecnorm{\vaidyaW_x^{1/2}y}{2} = \sqrt{\sum_{i=1}^\obs (\lev_{x, i}+\offset)\frac{(a_i^\top v)^2}{\slack_{x, i}^2}},
	\label{eq:def_local_norm}
\end{align}
and the ``slack sensitivity at $x$'' as
\begin{align}
	\thetaW_{x} \defn \parenth{\vecnorm{\frac{a_1}{\slack_{x, 1}}}{x}^2, \ldots, \vecnorm{\frac{a_\obs}{\slack_{x, \obs}}}{x}^2}\tp
	= \parenth{\frac{a_1^\top \vaidyaW_x^{-1}a_1}{\slack_{x, 1}^2} , \ldots, \frac{a_\obs^\top \vaidyaW_x^{-1}a_\obs}{\slack_{x, \obs}^2} }\tp.
	\label{eq:theta_Definition}
\end{align}
\end{subequations}

The following lemma provides useful properties of the leverage scores $\lev_{x}$~\eqref{eq:defn_lev_scores} and the slack sensitivity $\thetaW_{x}$ for all $x \in \intP$.
The importance of these properties is highlighted in the discussion that follows in the next subsection.
\begin{lemma}
	\label{lemma:first_bounds}
	For any $x \in \intP$, the following properties hold:
	\begin{enumerate}[label=(\alph*)]
		\item \label{item:sigma_bound} $\lev_{x, i} \in [0, 1]$ for all $i \in [\obs]$,
		% \item \label{item:g_bound} $g_{x, i} \in \brackets{\phiW, 1+\phiW}, i \in, [\obs]$;
		\item \label{item:theta_bound} $\thetaW_{x, i} \in \brackets{0, \sqrt{\obs/\dims}}$ for all $i \in [\obs]$, and
		% \item \label{item:theta_g_bound} $\thetaW_{x, i}g_{x, i} \in [0, 4\lev_{x, i}], i \in [\obs]$;
		\item \label{item:sigma_sum} $\sum_{i=1}^\obs\lev_{x, i}  = \dims$.
	\end{enumerate}
\end{lemma}
The proof of this lemma is provided in Section~\ref{ssub:proof_of_lemma_lemma:first_bounds}.
Next, we state a lemma that shows that if two points $x, y \in \intP$ are close in local norm, then the one-step transition probability distributions $\transition_x$ and $\transition_y$ of the random walk \VW{r} for suitable choice of the radius $r$, are also close in TV-distance.
\begin{lemma}
	\label{lemma:vaidya_walk_close_kernel}
	\begin{subequations}
	There exists a continuous non-decreasing function $f: [0, 1/12] \rightarrow \real_+$ with $f(1/12) \geq 1/9000$, such that
	for any $\epsilon \in (0, 1/4]$, the random walk \VW{r} with $r \in [0, f(\epsilon)]$ satisfies
	% the following is true.
	% Fix $\epsilon \in (0, 1/4]$ and choose the parameter $r$ for Vaidya walk smaller than $f(\epsilon)$.
	% Then
	\begin{align}
		\vecnorm{\proposal_x-\proposal_y}{\text{TV}} &\leq \epsilon, \quad \mbox{$\forall x, y \in \intP$ such that $\vecnorm{x-y}{x} \leq \displaystyle\frac{\epsilon r}{2(\obs\dims)^{1/4}}$ }
		,\quad \text{and}
		\label{eq:delta_p}\\
		\vecnorm{\transition_x-\proposal_x}{\text{TV}} &\leq 5\epsilon, \, \,\, \forall x \in \intP.
		\label{eq:delta_q_p}
	\end{align}
	\end{subequations}
\end{lemma}
We provide the proof of the lemma in Section~\ref{ssub:proof_of_lemma_lemma:vaidya_walk_close_kernel}.
We now provide an outline of the proof of Theorem~\ref{thm:mixing_time_bound} with some remarks that contrast Vaidya and Dikin walks.
% subsection some_auxiliary_results (end)


\subsection{Vaidya walk vs Dikin walk} % (fold)
\label{sub:vaidya_vs_dikin}
With the help of Lemma~\ref{lemma:Lovasz_theorem}, our proof of Theorem~\ref{thm:mixing_time_bound} proceeds by answering the following three questions:
\begin{enumerate}[label=(\alph*)]
	\item How does the local norm for the random walk relate to the cross-ratio?
	\item How small does the distance $\vecnorm{x-y}{x}$ need to be for the proposal distributions $\proposal_x$ and $\proposal_y$ to be close?
	\item What should be the value of $r$ such that for any $x \in \intP$, the Metropolis accept-reject step does not reject the proposal made by $\proposal_x$ with high probability?
\end{enumerate}
The first question is dealt directly in the proof of Theorem~\ref{thm:mixing_time_bound} while the other two questions are dealt by Lemma~\ref{lemma:vaidya_walk_close_kernel}.
We now provide a high-level discussion of these questions with remarks that contrast Vaidya walk with Dikin walk.
We contrast the two walks in terms of their \emph{respective} local-norms which are defined with respect to the matrix $\vaidyaW_x$ \eqref{eq:def_local_norm} for the Vaidya walk and with respect to the matrix $\hesslogbarr_x$ for the Dikin walk.

\paragraph{Part~(a):} % (fold)
\label{par:part_a}
While the cross-ratio is bounded below by a factor of $1/\sqrt{\dims}$ (see bound~\eqref{eq:hilbert_to_local}) to the local-norm for Vaidya walk, the same factor is $1/\sqrt{\obs}$ for Dikin walk (see, e.g., Lemma~9 in the paper~\cite{sachdeva2016mixing}).
Since the mixing time is affected inversely by this factor, we see that Vaidya walk has an \order{\sqrt{\obs/\dims}} advantage over Dikin walk for this part.
% paragraph part_a (end)

\paragraph{Part~(b):} % (fold)
\label{par:part_b}
To have a fast mixing chain, the distributions $\proposal_x$ and $\proposal_y$ need to be similar even when the points $x$ and $y$ are quite far apart.
If we apply Pinsker's inequality, it suffices to show that the covariance $\vaidyaW_y \vaidyaW_x^{-1} \approx \Ind_\dims$, when $x$ and $y$ are close in local-norm.
As we elaborate later, it suffices to control the eigenvalues of the matrix $\vaidyaW_y \vaidyaW_x^{-1}$ as a function of the local-norm of $x-y$.
We show that the local-norm associated with Vaidya walk is \emph{weaker} than the local-norm associated with Dikin walk, i.e., a small perturbation in local-norm changes the covariance matrix much more for Vaidya walk compared to Dikin walk.
As a result, $x$ and $y$ need to be much closer in local-norm for Vaidya walk (\order{1/(\obs\dims)^{1/4}}), compared to Dikin walk (\order{1/\sqrt{\dims}}) for $\proposal_x$ and $\proposal_y$ to be close.
Thus, on this front Vaidya walk is worse by a factor of \order{(\obs/\dims)^{1/4}} to Dikin walk.
% paragraph part_b (end)

\paragraph{Part~(c):} % (fold)
\label{par:part_c}
Note that the scale parameter in the proposal step of the walk \VW{r} is ${r}/{\parenth{\obs\dims}^{1/4}}$ and for Dikin walk the scale parameter is ${r'}/{\sqrt{\dims}}$, where $r$ and $r'$ are universal constants.
As a result, the scaling factor for Vaidya walk is worse by \order{(\obs/\dims)^{1/4}} to Dikin walk.
We now elaborate the reason for such a scaling at a high level.
The scaling factor is chosen such that the proposals are accepted with at least a constant probability for any $x \in \intP$.
In order to ensure this ``good proposal property'', we have to show that for the given scalings the proposal $z \sim \proposal_x$ satisfies the following two properties with high probability---$(i)$ the proposal $z$ remains inside $\Pspace$, and $(ii)$ the ratio $\density_z(x)/\density_x(z)$ is bounded away from zero.
It is easy to show part~$(i)$ using concentration of $\dims$-dimensional standard Gaussian vector on the unit sphere in $\realdim$, and the bound on the slack sensitivity $\theta_x$.
However, proving part~$(ii)$ needs some work that we summarize now.
To control the ratio $\density_z(x)/\density_x(z)$, we need to carefully bound the difference in the volume of unit-ellipsoids defined by $V_z$ and $V_x$, and the difference in local-norms at $z$ and $x$ scaled by $\sqrt{\obs\dims}$.
It turns out that controlling the eigenvalues of the matrix $\vaidyaW_z \vaidyaW_x^{-1}$ leads to a non-useful bound and hence we make use of Taylor expansion.
For both the parts $(i)$ and $(ii)$, it turns out that if the slack sensitivity $\theta_{x}$ is smaller, we can use a larger scaling factor.
Since the local sensitivity induced by Vaidya walk is larger than Dikin walk, we need to use a scaling of ${r/\parenth{\obs\dims}^{1/4}}$ for the former compared to $r'/\sqrt{\dims}$ for the later.
% paragraph part_c (end)

\paragraph{Who wins the race?} % (fold)
\label{par:who_wins_the_race_}
Putting the three parts together, we see that we gain a factor of \order{\sqrt{\obs/\dims}} in part(a) and lose a factor of \order{(\obs/\dims)^{1/4}} both in part~(b) and part~(c).
We prove later that the order of the ratio of mixing times of the two random walks is affected quadratically with respect to the gain from part~(a) times maximum of loss from parts~(b) and (c).
Thus we see an overall gain of \order{\sqrt{\obs/\dims}} in mixing time for Vaidya walk over Dikin walk.
% paragraph who_wins_the_race_ (end)

We are now ready to provide a formal proof of Theorem~\ref{thm:mixing_time_bound}.
% subsection discussion (end)

\subsection{Proof of Theorem~\ref{thm:mixing_time_bound}} % (fold)
\label{sub:proof_of_theorem_thm:mixing_time_bound}

We prove Theorem~\ref{thm:mixing_time_bound} using Lemmas~\ref{lemma:Lovasz_theorem}, \ref{lemma:first_bounds} and \ref{lemma:vaidya_walk_close_kernel}.
To invoke Lemma~\ref{lemma:Lovasz_theorem} for \VW{1/9000}, we need to show that for any two points $x, y \in \intP$ such that $d_\Pspace(x, y)$ is small, we have that $\vecnorm{\transition_x-\transition_y}{\text{TV}}$ is small.
Along the outline discussed in previous subsection, we break our analysis in two steps---(A) We first relate the cross-ratio $d_\Pspace(x, y)$ to the local norm \eqref{eq:def_local_norm} at $x$, and (B) then  use Lemma~\ref{lemma:vaidya_walk_close_kernel} to show that if $x, y \in \intP$ are close in local-norm, then the transition kernels $\transition_x$ and $\transition_y$ are close in TV-distance.

\paragraph{Step~(A):} % (fold)
\label{par:step_a}
We claim that for all $x, y \in \intP$, the cross-ratio can be lower bounded as
\begin{align}
	d_\Pspace(x, y) \geq \frac{1}{\sqrt{2\dims}} \vecnorm{x-y}{x}.
	\label{eq:hilbert_to_local}
\end{align}
Note that we have
\begin{align*}
	d_\Pspace(x, y) = \frac{ \vecnorm{e(x)-e(y)}{2} \vecnorm{x-y}{2} }{ \vecnorm{e(x)-x}{2} \vecnorm{e(y)-y}{2} }
	&\stackrel{(i)}{\geq} \max \braces{  \frac{\vecnorm{x-y}{2} }{\vecnorm{e(x)-x}{2}}, \frac{\vecnorm{x-y}{2} }{\vecnorm{e(y)-y}{2}}}\\
	&\stackrel{(ii)}{\geq} \max \braces{  \frac{\vecnorm{x-y}{2} }{\vecnorm{e(x)-x}{2}}, \frac{\vecnorm{x-y}{2} }{\vecnorm{e(y)-x}{2}}}
\end{align*}
where step $(i)$ follows from the inequality $ \vecnorm{e(x)-e(y)}{2} \geq \max\braces{\vecnorm{e(y)-y}{2}, \vecnorm{e(x)-x}{2}}$ and step $(ii)$ from the inequality $ \vecnorm{e(x)-x}{2} \leq {\vecnorm{e(y)-x}{2}}$.
Furthermore, from Figure~\ref{fig:cross_ratio}, we observe that
\begin{align}
	\max \braces{  \frac{\vecnorm{x-y}{2} }{\vecnorm{e(x)-x}{2}}, \frac{\vecnorm{x-y}{2} }{\vecnorm{e(y)-x}{2}}}
	= \max_{i \in [\obs]} \left\vert\frac{a_i^\top (x - y)}{\slack_{x, i}} \right\vert.
	\label{eq:relate_cross_ratio}
\end{align}
Note that maximum of a set of non-negative numbers is greater than the mean of the numbers.
Combining this fact with properties \ref{item:sigma_bound} and \ref{item:sigma_sum} from  Lemma~\ref{lemma:first_bounds},
\begin{align*}
	d_\Pspace(x, y) \geq \sqrt{ \frac{1}{\sum_{i=1}^\obs \parenth{\lev_{x, i}+ \offset}} \sum_{i=1}^\obs (\lev_{x, i}+ \offset) \frac{(a_i^\top(x-y))^2} {\slack_{x, i}^2} }  = \frac{\vecnorm{x-y}{x}}{\sqrt{2\dims}},
\end{align*}
thereby proving the claim~\eqref{eq:hilbert_to_local}.
% paragraph step_ A(end)

\paragraph{Step~(B):} % (fold)
\label{par:step_b}
By the triangle inequality, we have
\begin{align*}
	\vecnorm{\transition_x - \transition_y}{\text{TV}} \leq \vecnorm{\transition_x-\proposal_x}{\text{TV}} + \vecnorm{\proposal_x-\proposal_y}{\text{TV}} + \vecnorm{\proposal_y - \transition_y}{\text{TV}}.
\end{align*}
Thus, for any $(r, \epsilon)$ such that $\epsilon \in [0, 1/4]$ and $r \leq f(\epsilon)$, Lemma~\ref{lemma:vaidya_walk_close_kernel} implies that
\begin{align*}
	\vecnorm{\transition_x - \transition_y}{\text{TV}} \leq 11 \epsilon, \quad \mbox{$\forall x, y \in \intP$ such that $ \vecnorm{x-y}{x} \leq \displaystyle\frac{r\epsilon}{2 (\obs\dims)^{1/4}}$}.
\end{align*}
\\

\noindent Consequently, the walk \VW{r} satisfies the assumptions of Lemma~\ref{lemma:Lovasz_theorem} with
\begin{align*}
	\Delta \defn \frac{1}{\sqrt{2\dims}} \cdot \frac{r\epsilon} {2(\obs\dims)^{1/4}} \quad \text{ and }\quad \rho \defn 1-11\epsilon.
\end{align*}
Since $f(1/12) \geq 1/9000$, we can set $\epsilon = 1/12$ and $r = {1}/{9000}$, whence
\begin{align*}
	\Delta^2\rho^2 = \frac{(1-11\epsilon)^2\epsilon^2r^2}{8\dims\sqrt{\obs\dims}} = \frac{1}{12^4} \frac{1}{9000^2} \cdot\frac{1}{\dims\sqrt{\obs\dims}} \geq \frac{1}{2} \cdot 10^{-12} \, \frac{1}{\dims\sqrt{\obs\dims}}.
\end{align*}
Observing that $\Delta <1$ yields the claimed upper bound for the mixing time of Vaidya Walk.
% subsection proof_of_theorem_thm:mixing_time_bound (end)



% subsection proof_of_theorem_thm:central_start (end)

\subsection{Proof of Lemma~\ref{lemma:Lovasz_theorem}} % (fold)
\label{sub:proof_of_lemma_lemma:lovasz_theorem}
We begin by formally defining the conductance ($\conductance$) of a Markov chain on $(\Pspace, \mathbb{B}(\Pspace))$ with arbitrary transition kernel $\transition_x$ and stationary distribution $\stationary$
\begin{align*}
	\conductance
	\defn \inf_{\substack{\set \in \mathbb{B}(\Pspace)\\ \stationary(\set) \in (0, 1/2)}} \frac{\conductance(\set)}{\stationary(\set)}
	\quad \text{where} \quad
	\conductance(\set) \defn \int_\set \transition_u(\Pspace \cap \set^c) d\stationary(u)
	\quad \text{ for any } \set \subseteq \Pspace.
\end{align*}
The conductance denotes the measure of the flow from a set to its complement relative to its own measure, when initialized in the stationary distribution.
If the conductance is high, the following celebrated result shows that the Markov chain mixes fast.
\begin{lemma}
	\label{lemma:conductance_mixing}
	For any $M$-warm start $\initial$, the mixing time of the Markov chain with conductance~$\conductance$ is bounded as
	\begin{align*}
		\vecnorm{\initial\transition^k-\stationary}{\text{TV}} \leq \sqrt{M} \parenth{1-\frac{\conductance^2}{2}}^k \leq \sqrt{M} \exp\parenth{-k\frac{\conductance^2}{2} }.
	\end{align*}
\end{lemma}
Note that this result holds for a general distribution $\stationary$ although we apply  for uniform $\stationary$.
The result can be derived from Cheeger's inequality for continuous-space discrete-time Markov chain and elementary results in Calculus.
See, e.g., Theorem~1.4 and Corollary~1.5 in the paper~\cite{lovasz1993random} for a proof.
For ease in notation define $\Pspace \backslash \set \defn \Pspace \cap \set^c$.
We now state a key isoperimetric inequality.
\begin{lemma}[Theorem~6~\cite{lovasz1999hit}]
	\label{lemma:isoperimetry}
	For any measurable sets $\set_1, \set_2 \subseteq \Pspace$, we have
	\begin{align*}
		\vol(\Pspace \backslash \set_1  \backslash \set_2) \cdot  \vol(\Pspace)
		\geq d_\Pspace(\set_1, \set_2) \cdot \min\braces{\vol(\set_1), \vol(\set_2)},
	\end{align*}
	where $d_\Pspace(\set_1, \set_2) \defn \inf_{x \in \set_1, y \in \set_2} d_\Pspace(x, y)$.
\end{lemma}
(There is a typo on the RHS in the statement of the theorem in the paper.)
Since $\stationary$ is the uniform measure on $\Pspace$, this lemma implies that
\begin{align}
	\stationary(\Pspace \backslash \set_1  \backslash \set_2) \geq d_\Pspace(\set_1, \set_2) \min\braces{\stationary(\set_1), \stationary(\set_2)}.
	\label{eq:isoperimetry}
\end{align}
In fact, such an inequality holds for an arbitrary log-concave distribution~\cite{lovasz2003hit}.
In words, the inequality says that for a bounded convex set any two subsets which are far apart, can not have a large volume.
Taking these lemmas as given, we now complete the proof.

\begin{proof}[Proof of Lemma~\ref{lemma:Lovasz_theorem}]
	We first bound the conductance of the Markov chain using the assumptions of the lemma.
From Lemma~\ref{lemma:conductance_mixing}, we see that the Markov chain mixes fast if all the sets $\set$ have a high conductance $\conductance(\set)$.
We claim that
\begin{align}
	\label{eq:condutance_bound}
	\conductance \geq \frac{\rho \Delta}{16},
\end{align}
from which the proof follows by applying Lemma~\ref{lemma:conductance_mixing}.
We now prove the claim~\eqref{eq:condutance_bound} along the lines of Theorem~11 in the paper~\cite{lovasz1999hit}.
In particular, we show that under the assumptions in the lemma, the sets with bad conductance are far apart and thereby have a small measure under $\stationary$, whence the ratio $\conductance(\set)/\stationary(\set)$ is not arbitrarily small.
Consider a partition $\set_1, \set_2$ of the set $\Pspace$ such that $\set_1$ and $\set_2$ are measurable.
Since $\stationary$ is the uniform measure, it suffices to show that
\begin{align*}
	\int_{\set_1} \lazytrans_u(\set_2) du \geq  \frac{\rho \Delta}{16} \cdot \min\braces{\vol(\set_1), \vol(\set_2)}.
\end{align*}
Consider the sets
\begin{align}
	\set_1' &\defn \braces{ u \in \set_1 \bigg\vert \nolazytrans_u(\set_2) < \frac{\rho}{2} },\quad
	\set_2' &\defn  \braces{ v \in \set_2 \bigg\vert \nolazytrans_v(\set_1) < \frac{\rho}{2} }, \quad \text{and}\quad
	\set_3' &\defn \Pspace \backslash \set_1' \backslash \set_2'.
	\label{eq:define_three_sets}
\end{align}
If we have $\vol(\set_1') \leq \vol(\set_1)/2$ and consequently $\vol(\Pspace \backslash \set_1') \geq \vol(\set_1)/2$, then
\begin{align*}
	\int_{\set_1} \lazytrans_u(\set_2) du
	\stackrel{(i)}{\geq} \frac{1}{2} \int_{\Pspace \backslash \set_1'} \nolazytrans_u(\set_2) du
	\stackrel{(ii)}{\geq} \frac{\rho}{4}  \vol(\set_1)
	\stackrel{(iii)}{\geq} \frac{\rho\Delta}{16} \cdot \min\braces{\vol(\set_1), \vol(\set_2)},
\end{align*}
and we are done.
In the above sequence of inequalities, step $(i)$ follows from the definition of the kernel $\lazytrans$, step $(ii) $ follows from the definition of the set $\set_1'$~\eqref{eq:define_three_sets} and step $(iii)$ from the fact that $\Delta < 1 $.

Hence, without loss of generality we can assume $\vol(\set_i') \geq \vol(\set_i)/2$ for each $i\in \braces{1, 2}$.
Now for any $u \in \set_1'$ and $v \in \set_2'$ we have
\begin{align*}
	\vecnorm{\nolazytrans_u- \nolazytrans_v}{\text{TV}}
	\geq \nolazytrans_u(\set_1) - \nolazytrans_v(\set_1)
	= 1 -  \nolazytrans_u(\set_2) - \nolazytrans_v(\set_1)
	> 1 - \rho,
\end{align*}
and hence by assumption we have $d_\Pspace(\set_1', \set_2') \geq \Delta$.
Applying Lemma~\ref{lemma:isoperimetry} and the definition of $\set_3'$~\eqref{eq:define_three_sets} we find that
\begin{align}
	\vol(\set_3') \geq \Delta \min\braces{\vol(\set_1'), \vol(\set_2')}
	\geq \frac{\Delta}{2} \min\braces{\vol(\set_1), \vol(\set_2)}.
	\label{eq:iso_s3}
\end{align}
We now show that
\begin{align}
	\int_{\set_1} \lazytrans_u(\set_2)du = \int_{\set_2} \lazytrans_v(\set_1)dv.
	\label{eq:equality_of_T1_T2}
\end{align}
Noting that $\set_1 = \Pspace\backslash \set_2$, we have
\begin{align*}
	\frac{1}{\vol(\Pspace)} \int_{\set_1} \lazytrans_u(\set_2) du
	= \int_{\set_1} \lazytrans_u(\set_2) \stationarydensity(u) du
	= \conductance(\set_1)
	\stackrel{(i)}{=} \conductance(\Pspace\backslash \set_1)
	% &= \int_{\set_2} \nolazytrans_v(\set_1) \stationarydensity(v) dv\\
	= \frac{1}{\vol(\Pspace)} \int_{\set_2} \lazytrans_v(\set_1) dv,
\end{align*}
where in step $(i)$ we have used the following claim: For any measurable $\set \subseteq \Pspace$, we have
\begin{align}
	\conductance(\set) = \conductance(\Pspace\backslash \set).
	\label{eq:equality_of_conductance}
\end{align}
We now prove the claim~\eqref{eq:equality_of_conductance}.
Noting that $\int_{\Pspace} \lazytrans_u(\set) d\stationary(u) = \stationary(\set)$, we have
\begin{align*}
	\conductance(\Pspace\backslash \set)
	= \int_{\Pspace\backslash \set} \lazytrans_u(\set) d\stationary(u)
	= \int_{\Pspace} \lazytrans_u(\set) d\stationary(u) - \int_{\set} \lazytrans_u(\set) d\stationary(u)
	&= \stationary(\set) - \int_{\set} \lazytrans_u(\set) d\stationary(u).
\end{align*}
Using the fact that $1-\lazytrans_u(\set) = \lazytrans_u(\Pspace\backslash\set)$, we obtain
\begin{align*}
	\stationary(\set) - \int_{\set} \lazytrans_u(\set) d\stationary(u)
	= \int_{\set} d\stationary(u)  - \int_{\set} \lazytrans_u(\set) d\stationary(u)
	= \int_{\set}\lazytrans_u(\Pspace\backslash\set) d\stationary(u)
	= \conductance(\set),
\end{align*}
thereby yielding the claim.

Using the equation~\eqref{eq:equality_of_T1_T2}, we find that
\begin{align*}
	\int_{\set_1} \lazytrans_u(\set_2)	du
	= \frac{1}{2}\parenth{ \int_{\set_1} \lazytrans_u(\set_2) du + \int_{\set_2} \lazytrans_v(\set_1)dv }
	&\stackrel{(i)}{\geq} \frac{1}{2}\parenth{ \frac{1}{2} \int_{\Pspace \backslash \set_1'} \nolazytrans_u(\set_2) du
							 	+ \frac{1}{2} \int_{\Pspace \backslash \set_2'} \nolazytrans_v(\set_2) dv  }\\
	&\stackrel{(ii)}{\geq} \frac{\rho}{8} \vol(\set_3')\\
	&\stackrel{(iii)}{\geq} \frac{\rho\Delta}{16}  \min\braces{\vol(\set_1), \vol(\set_2)},
\end{align*}
where step $(i)$ follows from the definition of the kernel $\lazytrans$, step $(ii)$ follows from the definition of the set $\set_3'$~\eqref{eq:define_three_sets} and step $(iii)$ follows from the inequality~\eqref{eq:iso_s3}.
Putting together the pieces yields the claim~\eqref{eq:condutance_bound}.
\end{proof}

% subsection proof_of_lemma_lemma:lovasz_theorem (end)


\subsection{Proof of Lemma~\ref{lemma:first_bounds}} % (fold)
\label{ssub:proof_of_lemma_lemma:first_bounds}

To prove part~\ref{item:sigma_bound} observe that the Hessian $\hesslogbarr_x = \sum_{i=1}^\obs a_ia_i^\top/\slack_{x, i}^2$ is a sum of rank one positive semidefinite (PSD) matrices.
Also, we can write $\hesslogbarr_x = A_x\tp A_x$ where
\begin{align*}
	A_x \defn
	\begin{bmatrix}
		{a_1\tp}/{\slack_{x, 1}}\\
		\vdots\\
		{a_\obs\tp}/{\slack_{x, \obs}}
	\end{bmatrix} \quad \mbox{for all $x \in \intP$}.
\end{align*}
Since $\rank(A_x) = \dims$, we conclude that the matrix $\hesslogbarr_x$ is invertible and thus, both the matrices $\hesslogbarr_x$ and $\parenth{\hesslogbarr_x}^{-1}$ are PSD.
Since $\lev_{x, i} = a_i^\top \parenth{\hesslogbarr_x}^{-1} a_i / \slack_{x, i}^2$, we have $\lev_{x, i} \geq 0$.
Further, the fact that ${a_i a_i^\top}/{\slack_{x, i}^2} \preceq \hesslogbarr_x $ implies that   $\lev_{x, i} \leq 1$.

For part~\ref{item:sigma_sum}, using the equality $\trace(AB) = \trace(BA)$, we obtain
\begin{align*}
	\sum_{i=1}^\obs  \lev_{x, i}
	= \trace \parenth{\sum_{i=1}^\obs  \frac{a_i^\top \parenth{\hesslogbarr_x}^{-1} a_i}{\slack_{x, i}^2}}
	= \trace \parenth{\parenth{\hesslogbarr_x}^{-1}  \sum_{i=1}^\obs  \frac{a_i a_i^\top}{\slack_{x, i}^2}}
	= \trace (\Ind_\dims)
	= \dims.
\end{align*}

Now we prove part~\ref{item:theta_bound}.
Using the fact that $\lev_{x, i}\geq 0$, and an argument similar to part~\ref{item:sigma_bound} we find that that the matrices $\vaidyaW_x$ and $\vaidyaW_x^{-1}$ are PSD.
Since $\thetaW_{x, i} = {a_i^\top \vaidyaW_x^{-1} a_i}/{\slack_{x, i}^2}$, we have $\thetaW_{x,i}\geq 0$.
It is straightforward to see that $\offset\hesslogbarr_x \preceq \vaidyaW_x$ which implies that $\thetaW_{x, i} \leq \lev_{x, i}/\beta$.
Further, we also have $\parenth{\lev_{x, i}+ \offset}\frac{a_i a_i^\top}{\slack_{x, i}^2} \preceq \vaidyaW_x$ and whence $ \thetaW_{x, i} \leq 1/\parenth{\lev_{x, i}+ \offset} $.
Combining the two inequalities yields the claim.

% subsection proof_of_lemma_ppt:sigma_and_theta_bounds (end)


\subsection{Proof of Lemma~\ref{lemma:vaidya_walk_close_kernel}} % (fold)
\label{ssub:proof_of_lemma_lemma:vaidya_walk_close_kernel}

We prove the lemma for the following function
\begin{subequations}
\begin{align}
	f(\epsilon) \defn 
	\min\braces{
	\frac{1}{40\sqrt{\log\parenth{4/\epsilon}}},
	\frac{\sqrt{\epsilon}}{30\sqrt{\log\parenth{4/\epsilon}}},
	\frac{\epsilon}{84 \cdot \parenth{\log(4/\epsilon)}^{\frac{3}{2}}}, 
	\frac{\sqrt{\epsilon}}{100 \cdot {\log(4/\epsilon)}} 
	}. 
	\label{eq:r_2_epsilon}
\end{align}
% \begin{align}
% 	r_1(\epsilon) &\defn \min \braces{\frac{1}{20 \sqrt{1+\sqrt{2}\log\parenth{4/\epsilon}}}, \frac{\epsilon}{\sqrt{72 \log\parenth{2/\epsilon}}}, \sqrt{\frac{\epsilon}{200\sqrt{3} e \log\parenth{4/\epsilon}}}},\quad\text{and} \label{eq:r_1_epsilon}\\
% 	r_2(\epsilon) &\defn \min\braces{\frac{1}{20\sqrt{1 + \sqrt{2} \log\parenth{4/\epsilon}}}, \frac{\epsilon}{7\sqrt{15}\parenth{2e/3 \cdot \log(4/\epsilon)}^{\frac{3}{2}}}, \sqrt{\frac{\epsilon}{100\sqrt{105}\parenth{e \log(4/\epsilon)}^2}}}. \label{eq:r_2_epsilon}
% \end{align}
\end{subequations}
Some straightforward algebra shows that $f(1/12) \geq 1/9000$.


\subsubsection{Proof of claim~\eqref{eq:delta_p}} % (fold)
\label{sssub:proof_of_claim_eq:delta_p}
In order to bound the total variation distance $\vecnorm{\proposal_x-\proposal_y}{\text{TV}}$, we apply Pinsker's inequality, which provides an upper bound on the TV-distance in terms of the KL Divergence:
\begin{align*}
	\vecnorm{\proposal_x-\proposal_y}{\text{TV}} \leq \sqrt{2\text{KL}(\proposal_x \Vert \proposal_y)}.
\end{align*}
For Gaussian distributions, the KL Divergence has a closed form expression.
In particular, for two normal-distributions $\mathcal{G}_1 = \NORMAL\parenth{\mu_1, \levmatrix_1}$ and $\mathcal{G}_2 = \NORMAL\parenth{\mu_2, \levmatrix_2}$, the Kullback-Leibler divergence between the two is given by
    \begin{align*}
        \text{KL}(\mathcal{G}_1\Vert \mathcal{G}_2) = \frac{1}{2} \Big({
        \trace\parenth{\levmatrix_1^{-1}\levmatrix_2}
        -\dims
        +\log \det \parenth{\levmatrix_1\levmatrix_2^{-1}}
        + \parenth{\mu_1-\mu_2}^\top \levmatrix_1^{-1}\parenth{\mu_1-\mu_2}
        }\Big).
    \end{align*}
Substituting $\mathcal{G}_1 = \proposal_x$ and $\mathcal{G}_2 = \proposal_y$ into the above expression and applying Pinsker's inequality, we find that
\begin{align}
	\vecnorm{\proposal_x-\proposal_y}{\text{TV}}^2 \leq 2 \text{KL}( \proposal_y \Vert \proposal_x )
	&=  \trace(\vaidyaW_y\vaidyaW_x^{-1}) - \dims + \log\parenth{\frac{\det \vaidyaW_x}{\det \vaidyaW_y}} + \frac{\sqrt{\obs \dims}}{r^2} \vecnorm{x-y}{x} \notag \\
	&= \sum_{i=1}^\dims \parenth{\lambda_i - 1 + \log\frac{1}{\lambda_i}} + \frac{\sqrt{\obs \dims}}{r^2} \vecnorm{x-y}{x},
	\label{eq:final_KL_expression}
\end{align}
where $\lambda_1, \ldots, \lambda_\dims >0$ denote the eigenvalues of $\vaidyaW_y \vaidyaW_x^{-1}$
and we have used the facts that $\det\parenth{\vaidyaW_y \vaidyaW_x^{-1}} = \prod_{i=1}^\dims \lambda_i$ and $\trace\parenth{\vaidyaW_y \vaidyaW_x^{-1}} = \sum_{i=1}^\dims \lambda_i$.
The following lemma is useful to bound the expression~\eqref{eq:final_KL_expression}.
\begin{lemma}
    \label{lemma:close_hessian_eigenvalues}
    For any scalar $t \in [0, \sqrt{\dims}/12]$ and any pair $x, y \in \intP$ such that \mbox{$\vecnorm{x-y}{x} \leq t / (\obs\dims)^{1/4}$} we have
    \begin{align*}
    	\parenth{1 - \frac{6 t}{\sqrt{\dims}} + \frac{ t^2}{\dims}}\vaidyaW_x \leq \vaidyaW_y \leq \parenth{1 + \frac{6 t}{\sqrt{\dims}} + \frac{t^2}{\dims}}\vaidyaW_x.
    \end{align*}
\end{lemma}
See Appendix~\ref{sub:proof_of_lemma_lemma:close_hessian_eigenvalues} for its proof.
For $\epsilon \in (0, 1/4]$ and $r = 1/9000$, we have $t = \epsilon r /2 \leq 1/12$, whence the eigenvalues $\{\lambda_i, i \in [\dims]\}$  can be sandwiched as
\begin{align}
	1 - \frac{3 \epsilon r}{\sqrt{\dims}} + \frac{\epsilon^2r^2}{4\dims}
	\leq \lambda_i
	\leq 1 + \frac{3 \epsilon r}{\sqrt{\dims}} + \frac{\epsilon^2r^2}{4\dims} \quad \mbox{for all $i \in \dims$}.
\label{eq:lambda_bounds}
\end{align}
We are now ready to bound the TV distance between $\proposal_x$ and $\proposal_y$.
Using the bound~\eqref{eq:final_KL_expression} and the inequality $\log \gamma \leq \gamma - 1$ , valid for $ \gamma>0$, we obtain
\begin{align*}
	\vecnorm{\proposal_x-\proposal_y}{\text{TV}}^2
	\leq \sum_{i=1}^\dims \parenth{\lambda_i - 2 + \frac{1}{\lambda_i}} + \frac{\sqrt{\obs \dims}}{r^2} \vecnorm{x-y}{x}.
\end{align*}
Using the assumption that $\vecnorm{x-y}{x}\leq {\epsilon r}/\parenth{2(\obs\dims)^{1/4}}$, and plugging in the bounds~\eqref{eq:lambda_bounds} for the eigenvalues $\{\lambda_i, i \in [\dims]\}$, we find that
\begin{align*}
	\sum_{i=1}^\dims \parenth{\lambda_i - 2 + \frac{1}{\lambda_i}} + \frac{\sqrt{\obs \dims}}{r^2} \vecnorm{x-y}{x}
	\leq \frac{141 \epsilon^2 r^2}{4\dims} + \frac{\epsilon^2}{4}.
\end{align*}
In asserting this inequality, we have used the facts that
\begin{align*}
	\frac{1}{1 - 6\gamma + \gamma^2} \leq 1 + 6 \gamma + 70 \gamma^2,
	\quad \text{and}\quad
	\frac{1}{1 + 6\gamma + \gamma^2} \leq 1 - 6\gamma + 70 \gamma^2
	\quad \mbox{for all $\gamma \in \brackets{0, \frac{1}{12}}$}.
\end{align*}
Note that for any  $r \in [0, 1/12]$ we have that $141 r^2/(4\dims) \leq 1/2$.
Putting the pieces together yields $\vecnorm{\proposal_x-\proposal_y}{\text{TV}} \leq \epsilon$, as claimed.

%%%%%%%%%      ANOTHER WAY TO WRITE THE PREVIOUS PARA    %%%%%%%%%%
% We are now ready to bound the TV distance between $\proposal_x$ and $\proposal_y$:
% \begin{align*}
% 	\vecnorm{\proposal_x-\proposal_y}{\text{TV}}^2
% 	&\stackrel{(i)}{\leq} \sum_{i=1}^\dims \parenth{\lambda_i - 2 + \frac{1}{\lambda_i}} + \frac{\sqrt{\obs \dims}}{r^2} \vecnorm{x-y}{x}
% 	\stackrel{(ii)}{\leq} \frac{71 \epsilon^2 r^2}{4\dims} + \frac{\epsilon^2}{4}
% 	\stackrel{(iii)}{\leq} \epsilon^2,
% \end{align*}
% where step $(i)$ follows from the inequality $\log \gamma \leq \gamma - 1$ for $ \gamma>0$.
% For step $(ii)$, we have used the  and the following observations: (a) The map $\gamma \mapsto \gamma + \frac{1}{\gamma}-2$ is convex for $\gamma > 0$,
% and (b) for $0 \leq \gamma \leq \frac{1}{12}$ we have
% \begin{align*}
% 	\frac{1}{1 - 6\gamma + \gamma^2} \leq 1 + 6 \gamma + 70 \gamma^2,
% 	\quad \text{and}\quad
% 	\frac{1}{1 + 6\gamma + \gamma^2} \leq 1 - 6\gamma + 70 \gamma^2.
% \end{align*}
% Finally, in step $(iii)$ we have used the fact that for any  $r \in [0, 1/6]$ we have that $71 r^2/(4\dims) \leq 1/2$.
% Thus we obtain $\vecnorm{\proposal_x-\proposal_y}{\text{TV}} \leq \epsilon$, as claimed.
%% subsubsection proof_of_claim_eq:delta_p (end)
% paragraph proof_of_claim_eq:delta_p (end)

\subsubsection{Proof of claim~\eqref{eq:delta_q_p}} % (fold)
\label{ssub:proof_of_claim_eq:delta_q_p}
Note that
\begin{align}
	\transition_x(\braces{x}) = \proposal_x(\Pspace^c) + 1-\int_\Pspace \min\braces{1, \frac{\density_z(x)}{\density_x(z)}} \density_x(z) dz,
	\label{eq:remain_at_x}
\end{align}
where $\Pspace^c$ denotes the complement of $\Pspace$.
Consequently, we obtain that
\begin{align}
	\vecnorm{\proposal_x - \transition_x}{\text{TV}}
	&= \frac{1}{2} \parenth{\transition_x(\braces{x}) + \int_{\realdim} \density_x(z) dz - \int_{\Pspace} \min\braces{1, \frac{\density_z(x)}{\density_x(z)}} \density_x(z) dz} \notag\\
	% &= \frac{1}{2} \parenth{\proposal_x(\Pspace^c) + 2 -  2 \int_{\Pspace} \min\braces{1, \frac{\density_z(x)}{\density_x(z)}} \density_x(z) dz}\\
	&= \frac{1}{2} \parenth{\proposal_x(\Pspace^c) + 2 -  2 \int_{\realdim} \min\braces{1, \frac{\density_z(x)}{\density_x(z)}} \density_x(z) dz + 2 \int_{\Pspace^c} \min\braces{1, \frac{\density_z(x)}{\density_x(z)}} \density_x(z) dz }\notag\\
	&\leq \underbrace{\frac{3}{2} \proposal_x(\Pspace^c)}_{T_1\defn} + \underbrace{1-\Exs_{z \sim \proposal_x} \brackets{\min\braces{1, \frac{\density_z(x)}{\density_x(z)}}}}_{T_2\defn},
	\label{eq:bound_on_jumping_out}
\end{align}
Thus it suffices to show that both $T_1$ and $T_2$ are small, where the probability is taken over the randomness in the proposal $z$.
In particular, we show that $T_1 \leq \epsilon$ and $T_2 \leq 4\epsilon$.
\\

\noindent \textbf{Bounding $T_1$:}
Note that a random variable $z \sim \proposal_x$ can be written as
\begin{align}
    z \stackrel{d}{=} x + \frac{r}{(\obs\dims)^{1/4}} \vaidyaW_x^{-{1}/{2}} \rvg,
    \label{eq:z_x_relation}
\end{align}
where  $\rvg \sim \NORMAL\parenth{0, \Ind_\dims}$ and $\stackrel{d}{=}$ denotes equality in distribution.
Using equation~\eqref{eq:z_x_relation} and definition~\eqref{eq:theta_Definition} of $\thetaW_{x, i}$, we obtain that
\begin{align}
 \frac{\parenth{a_i^\top \parenth{z-x}}^2 }{\slack_{x, i}^2}
 	& = \frac{r^2}{\parenth{\obs \dims}^{\frac{1}{2}}} \brackets{\frac{a_i^\top \vaidyaW_x^{-{1}/{2}} \rvg }{\slack_{x, i}}}^2
 	\stackrel{(i)}{\leq} \frac{r^2}{\parenth{\obs \dims}^{\frac{1}{2}}} \thetaW_{x,i} \vecnorm{\rvg}{2}^2
 	\stackrel{(ii)}{\leq} \frac{r^2}{\dims} \vecnorm{\rvg}{2}^2,
 \label{eq:bound_on_deviation}
\end{align}
where step $(i)$ follows from Cauchy-Schwarz inequality and step $(ii)$ from the bound on $\thetaW_{x, i}$ from Lemma~\ref{lemma:first_bounds}\ref{item:theta_bound}.
% For random variables $g$ and $z$ distributed as $g \sim \NORMAL\parenth{0, \Ind_\dims}$ and $z \sim \proposal_x$,
Define the events
\begin{align*}
	\mathcal{E} \defn \braces{\frac{r^2}{\dims} \vecnorm{\rvg}{2}^2 < 1}
	\quad \text{and} \quad
	\mathcal{E}'\defn \braces{z \in \intP}.
\end{align*}
Inequality~\eqref{eq:bound_on_deviation} implies that $\mathcal{E} \subseteq \mathcal{E}'$ and hence $\Prob\brackets{\mathcal{E}'} \geq \Prob\brackets{\mathcal{E}}$.
Using a standard Gaussian tail bound and noting that $r \leq \frac{1}{1 + \sqrt{2/\dims \log (2/\epsilon)}}$, we obtain
$ \Prob\brackets{\mathcal{E}} \geq 1-\epsilon/2$ and whence $ \Prob\brackets{\mathcal{E}'} \geq 1-\epsilon/2$.
Thus, we have shown that $\Prob\brackets{z \notin \Pspace} \leq \epsilon/2$ which implies that $T_1 \leq \epsilon$.
\\

\noindent \textbf{Bounding $T_2$:}
By Markov's inequality, we have
\begin{align}
	\Exs_{z \sim \proposal_x} \brackets{\min\braces{1, \frac{\density_z(x)}{\density_x(z)}}} \geq \alpha \Prob\brackets{\density_z(x) \geq \alpha \density_x(z)}
	\quad \mbox{for all $\alpha \in [0, 1]$}. \label{eq:markov_inequality}
\end{align}
By definition~\eqref{eq:proposal_density} of $\density_x$, we obtain
\begin{align*}
	\frac{\density_z(x)}{\density_x(z)} = \exp \parenth{-\frac{\sqrt{\obs\dims}}{2r^2} \parenth{\vecnorm{z-x}{z}^2- \vecnorm{z-x}{x}^2 }+ \frac{1}{2} \parenth{\log \det \vaidyaW_z - \log \det \vaidyaW_x}}.
\end{align*}
The following lemma provides us with useful bounds on the two terms in this expression.
 \begin{lemma}
 	\label{lemma:change_in_log_det_and_local_norm}
 	For any $\epsilon \in (0, \frac{1}{4}]$ and $x \in \intP$, if $r \leq f(\epsilon)$, we have
 	\begin{subequations}
 		\begin{align}
 			 \Prob_{z\sim \proposal_x}\brackets{\frac{1}{2}\log\det \vaidyaW_z - \frac{1}{2}\log\det \vaidyaW_x \geq -\epsilon } &\geq 1 - \epsilon,
 			 \ \text{and}\label{eq:whp_log_det_filter}\\
 			 \Prob_{z\sim \proposal_x}\brackets{\vecnorm{z - x}{z}^2 - \vecnorm{z - x}{x}^2 \leq 2 \epsilon \frac{r^2}{\sqrt{\numobs\usedim}} } &\geq 1 - \epsilon \label{eq:whp_local_norm}.
 		\end{align}
 	\end{subequations}
 \end{lemma}
The proof of this lemma is provided in Appendix~\ref{sec:proof_of_lemma_lemma:change_in_log_det_and_local_norm}.
Using Lemma~\ref{lemma:change_in_log_det_and_local_norm}, we now complete the proof.
For $r \leq f(\epsilon)$, we obtain $\density_z(x) /\density_x(z) \geq \exp\parenth{-2\epsilon} \geq 1- 2\epsilon$ with probability at least $1-2\epsilon$.
Substituting $\alpha = 1-2\epsilon$ in inequality~\eqref{eq:markov_inequality} yields the bound $T_2 \leq 4\epsilon$.
% subsubsection proof_of_claim_eq:delta_q_p (end)

% subsection proof_of_lemma_lemma:vaidya_walk_close_kernel (end)



